\documentclass{beamer}
\setbeamertemplate{navigation symbols}{}

\usepackage[polish]{babel}
\usepackage[utf8]{inputenc}
\usepackage{polski}
\usepackage[T1]{fontenc}
\usepackage{hyperref}

\usepackage{amsmath}
\usepackage{amsfonts}
\usepackage{amssymb}
\usepackage{amsthm}

\usetheme{Darmstadt}

\beamersetuncovermixins{\opaqueness<1>{25}}{\opaqueness<2->{15}}
\begin{document}
\title{Interfejs kamery oraz projektora}  
\author{Piotr Róż \\ Jacek Zawistowski}
\institute
{
  Politechnika Warszawska\\
  Wydział Elektroniki i Technik Informacyjnych\\
  Instytut Systemów Elektronicznych
  \and
  Politechnika Warszawska\\
  Wydział Elektroniki i Technik Informacyjnych\\
  Intytut ?
}
\date{\today} 

\begin{frame}
\titlepage
\end{frame}


\section{Sekwencja nukleotydowa ogórka}
\subsection{Proces sekwencjonowania}
% http://www.twojeinnowacje.pl/polacy-zsekwencjonowali-genom-ogorka-i-dowiedli-jak-rosliny-reaguja-na-stres
% http://www.polskieradio.pl/23/266/Artykul/432667,Polacy-zsekwencjonowali-genom-ogorka
% http://www.sggw.pl/2011/08/30/genom-ogorka-opublikowany/
\begin{frame}
	\begin{columns}
		\begin{column}{4cm}
			\begin{figure}[htb]
	  			\begin{center}
	  				\includegraphics[angle=0,scale=0.4]{img/ogorek.jpg}
	  			\end{center}
	        \end{figure}
    	\end{column}
    	\begin{column}{8cm}
  			\begin{itemize}
    			\item Praca zespołu prof. Zbigniewa Przybeckiego
    			\item Pierwszy tak skomplikowany genom zsekwencjonowany przez 'polski' zespół (20-osobowy)
    			\item Porównanie z genomem zsekwencjonowanym przez naukowców z Chin
    			\item Wyścig
    			\item Wyniki badań:
    			\begin{itemize}
    				\item DNA obu roślin bardzo się różnią
    				\item Wpływ klimatu
    				\item Zdolność adaptacyjna roślin
    				\item Ogromne ilości danych
    				\item Konieczność przeprowadzenia łatwej wizualizacji genomu
    			\end{itemize}
  			\end{itemize}
  		\end{column}
	\end{columns}
\end{frame}


\subsection{Baza zebranej wiedzy}
\begin{frame}\frametitle{Dane zebrane w postaci tabeli...są trudne do przeglądania.}
	\begin{figure}[htb]
		\begin{center}
	  		\includegraphics[angle=0,scale=0.3]{img/dane.jpg}
		\end{center}
	\end{figure}
\end{frame}


\section{Przeglądarka genomu}
\subsection{Rozwiązanie}
\begin{frame}
	\begin{columns}
		\begin{column}{6cm}
			\begin{figure}[htb]
	  			\begin{center}
	  				\includegraphics[angle=0,scale=0.4]{img/sekwencja.jpg}
	  			\end{center}
	        \end{figure}
    	\end{column}
    	\begin{column}{6cm}
			Z powodu istnienia ogromnej ilości generowanych danych, zaistniała konieczność wykorzystania:
			\begin{itemize}
				\item Praktycznie nieograniczonej mocy obliczeniowej komputerów
				\item Statystyki
				\item Bioinformatyki
			\end{itemize}
			w celu zoptymalizowania procesu ich przeglądania oraz przetwarzania.
		\end{column}
	\end{columns}
\end{frame}


\subsection{Rice Genome Browser}
\begin{frame}\frametitle{Przeglądarka genomu ryżu}
	\begin{figure}[htb]
		\begin{center}
	  		\includegraphics[angle=0,scale=0.6]{img/rice_logo.jpg}
	  	\end{center}
	\end{figure}
	
	\begin{figure}[htb]
		\begin{center}
	  		\includegraphics[angle=0,scale=0.6]{img/rice_chr.jpg}
	  	\end{center}
	\end{figure}
	
	\url{http://rice.plantbiology.msu.edu/cgi-bin/gbrowse/rice/}
\end{frame}


\begin{frame}\frametitle{Przeglądarka genomu ryżu}
	\begin{figure}[htb]
		\begin{center}
	  		\includegraphics[angle=0,scale=0.32]{img/rice_main.jpg}
	  	\end{center}
	\end{figure}
\end{frame}

\subsection{GBrowse}
\begin{frame}\frametitle{Algorytmy i struktury danych}
	\textbf{GBrowse} jest graficznym interfejsem wyświetlającym informacje o genomie począwszy od pojedynczego nukleotydu, a skończywszy na strukturze całego chromosomu.

	\begin{block}{Wyświetlane struktury}
		\begin{itemize}
			\item Scaffoldy
			\item Contigi
			\item Markery
			\item ...
		\end{itemize}
	\end{block}
\end{frame}

\subsection{GBrowse}
\begin{frame}\frametitle{Algorytmy i struktury danych}
	\begin{block}{Interval Tree}
		Drzewo przedziałowe:
		\begin{itemize}
			\item Każdy węzeł przetrzymuje informację o przedziale domkniętym
			\item Wyszukiwanie przedziałów, mających z zadanym przedziałem niepuste części wspólne
			\item Wydajność (wygenerowanie 5000 widoków - ramek):
				\begin{itemize}
					\item Standardowe podejście wykonujące odpowiednie operacje na bazie danych: \\1183.99100018s
					\item Podejście opierające się o drzewo przedziałowe: \\187.058000088s
				\end{itemize}
		\end{itemize}
	\end{block}
\end{frame}


\begin{frame}\frametitle{Struktura aplikacji}
	\begin{columns}
		\begin{column}{6cm}
			\begin{figure}[htb]
				\begin{center}
	  				\includegraphics[angle=0,scale=0.25]{img/architektura.pdf}
	  			\end{center}
			\end{figure}
    	\end{column}
    	\begin{column}{6cm}
    		\begin{block}{Zalety}
    			\begin{itemize}
    				\item Multiplatformowość
    				\item Multisystemowosć
    				\item Bezpieczeństwo
    				\item Brak obciążenia komputera klienta
    			\end{itemize}
    		\end{block}
    		\begin{block}{Wady}
    			\begin{itemize}
    				\item Spadek wydajności przy dużym obciążeniu serwera
    			\end{itemize}
    		\end{block}
		\end{column}
	\end{columns}	
\end{frame}


\begin{frame}\frametitle{Prototyp aplikacji}
	\begin{figure}[htb]
		\begin{center}
	  		\includegraphics[angle=0,scale=0.45]{img/chromosomy.jpg}
	  	\end{center}
	\end{figure}
\end{frame}


\begin{frame}\frametitle{Prototyp aplikacji}
	\begin{figure}[htb]
		\begin{center}
	  		\includegraphics[angle=0,scale=0.35]{img/chromosom3.jpg}
	  	\end{center}
	\end{figure}
\end{frame}


\subsection{Podsumowanie}
\begin{frame}\frametitle{Podsumowanie}
	\begin{itemize}
		\item Czasochłoonna obróbka danych
		\item Konieczna maksymalna elastyczność aplikacji
		\item Duże znaczenie przeglądarki
	\end{itemize}
	\begin{block}{Plany na przyszłość}
		\begin{itemize}
			\item Zgromadzenie wszystkich danych
			\item Wyświetlenie wszystkich struktur
			\item Implementacja algorytmu (?)
		\end{itemize}
	\end{block}
\end{frame}


\begin{frame}\frametitle{Dziękuję}
\begin{center}
  \Huge{\textbf{Dziękuję za uwagę!}}
\end{center}
\end{frame}

\end{document}