\documentclass{beamer}
\setbeamertemplate{navigation symbols}{}

\usepackage[polish]{babel}
\usepackage[utf8]{inputenc}
\usepackage{polski}
\usepackage[T1]{fontenc}
\usepackage{hyperref}

\usepackage{amsmath}
\usepackage{amsfonts}
\usepackage{amssymb}
\usepackage{amsthm}

\usetheme{Darmstadt}

\beamersetuncovermixins{\opaqueness<1>{25}}{\opaqueness<2->{15}}
\begin{document}
\title{Interfejs kamery oraz projektora \\ Projekt z przedmiotu SWIZ}  
\author{Piotr Róż \\ Jacek Zawistowski}
\institute
{
  Politechnika Warszawska\\
  Wydział Elektroniki i Technik Informacyjnych\\
  Instytut Systemów Elektronicznych
  \and
  Politechnika Warszawska\\
  Wydział Elektroniki i Technik Informacyjnych\\
  Intytut ?
}
\date{\today} 

\begin{frame}
\titlepage
\end{frame}

\section{Wstęp}
\begin{frame}\frametitle{Wstęp}
  \begin{block}{Motywacja}
    \begin{itemize}
     \item Wsparcie nowego laboratorium potrzebnym oprogramowaniem; \pause
     \item Możliwość pozostawienia po sobie śladu na uczelni; \pause
     \item ...?
    \end{itemize}
  \end{block}
\end{frame}

\begin{frame}\frametitle{Wstęp}
  \begin{block}{Cel pracy}
    \begin{itemize}
     \item Utworzenie interfejsu do zestawu:
      \begin{itemize}
       \item Projektor;
       \item Kamera. 
      \end{itemize}\pause
     \item Implementacja podstawowych operacji na obrazie:
      \begin{itemize}
       \item Dodawanie;
       \item Odejmowanie;
       \item Mnożenie;
       \item Itd.
      \end{itemize}
    \end{itemize}
  \end{block}
\end{frame}

\section{Architektura}
\subsection{Architektura systemu}
\begin{frame}\frametitle{Schemat}
  Tutaj jakiś obrazek pokazujący jak to wszystko wygląda połączeniowo
\end{frame}

\begin{frame}\frametitle{Sekwencja działań}
  \begin{columns}
    \begin{column}{4cm}
      Obrazek z poprzedniego slajdu jakiś mniejszy z boku
    \end{column}
    
    \begin{column}{8cm}
      \begin{enumerate}
       \item Wczytanie obrazka przez użytkownika, w celu wyświetlenia go na projektorze;\pause
       \item Ustawienie parametrów wyświetlania;\pause
       \item Zapisanie sekwencji obrazów do plików;\pause
       \item Odczytanie wyświetlanego obrazu obiektu przez kamerę;\pause
       \item Zapisanie sekwencji obrazów z kamery;\pause
       \item Możliwość wykonania operacji na obrazach.
      \end{enumerate}
    \end{column}
  \end{columns}
\end{frame}

\subsection{Architektura oprogramowania}
\begin{frame}\frametitle{Schemat}
  Tutaj znowu jakiś ładny rysunek z bibliotekami itd
  Pod spodem opisane biblioteki - po co są użyte.
\end{frame}

\section{Podsumowanie}
\begin{frame}\frametitle{Podsumowanie}
  \begin{alertblock}{Możliwe problemy}
    \begin{itemize}
     \item Synchronizacja obrazów z projektowa i kamery;
     \item ?
    \end{itemize}
  \end{alertblock}
\end{frame}

\begin{frame}\frametitle{Dziękujemy}
\begin{center}
  \Huge{\textbf{Dziękujemy za uwagę!}}
\end{center}
\end{frame}

\end{document}